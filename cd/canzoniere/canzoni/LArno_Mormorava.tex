\section{L'Arno mormorava}
{\setstretch{1.2}\fontsize{13}{16}\selectfont
\begin{canzone}
L'Arno mormorava
calmo e placido di fronte
ai primi allievi in marcia sopra il ponte.
L'esercito avanzava,
trasportava i propri scudi,
mostrando al vento i fieri petti ignudi.

Carrelli ovunque pien di gavettoni!
Andavano a lavar quei gran puzzoni!

S'udiva intanto dalle vie d'intorno
il forte canto ed il suonar del corno.
Era il gran grido delle invitte schiere.

E l'Arno mormorò:
"Non passa l'ingegnere!"

Ma dopo la battaglia
si parlò di codardia,
ch'era il nemico presto corso via.
Saliva la marmaglia
sulle scale in Cavalieri,
dava una foto in pasto a dei ciarlieri.

Giornali scrive certo non la storia
il santannino che grida vittoria.

Veniva intanto in mente a ogni buffone
il tempo dell'esame d'ammissione.
Credeva d'esser genio di sapere.

Ma il test comandò:
"Segato è l'ingegnere!"

E ritornò una sera
in San Francesco la tenzone:
si miser tutti quanti in formazione.
S'ergeva tra la schiera
maestoso un gran fortino:
sottrasse ogni speranza al santannino!

Carrelli, ultrà e scudi ancor più grossi
e il fuoco e i calici scottanti e rossi!

Cercò la feccia allor di rimediare
Le scale in Carovana di occupare
Ma i Normalisti eran lì a tenere.

La piazza comandò:
"Indietro va', ingegnere!"

E indietreggiò sconfitto
fino in Santa Caterina,
si chiuse nella sua volgar latrina.
Placatosi il conflitto
tra le schiere il Costruttore
apparve assieme al nostro Gran Priore!

Sancì allora questa sacra vista
la nobile vittoria normalista!

Sicura casa e libere le scale
e tacque l'Arno e si zittì il giornale.
Sul patrio suolo vinti gl'ingegneri,

la scienza non trovò
spazzini e parrucchieri!
\end{canzone}
}
